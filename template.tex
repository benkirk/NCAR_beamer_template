\documentclass[aspectratio=1610]{beamer}

% LaTeX Packages
\usepackage{layout}
\usepackage{relsize}
\usepackage{lipsum}
%\usepackage[brandedfonts]{ncar_branding} % initialize ncar_branded package here, perhaps with options, or beamer will do it later

% Beamer Specfics
%\useinnertheme[shadow=true]{ncarrounded} % if you prefer rounded boxes, with or without shadows
\usetheme{ncar}

\usefonttheme[onlymath]{serif}
%\usefonttheme{serif}



\begin{document}

%%%%%%%%%%%%%%%%%%%%%%%%%%%%%%%%%%%%%%%%%%%%%%%%%%%
\title{This is my title}
\subtitle{and I might have a subtitle too!}
\author{Author One, Author Two}
\begin{NCARtitleframe}%[common/images/wallpaper/NCAR_Clouds.jpg]
  %\pgfsetfillopacity{0.85}
  \vspace{6em}
  \maketitle
\end{NCARtitleframe}


%%%%%%%%%%%%%%%%%%%%%%%%%%%%%%%%%%%%%%%%%%%%%%%%%%%
%%%%%%%%%%%%%%%%%%%%%%%%%%%%%%%%%%%%%%%%%%%%%%%%%%%
\begin{frame}
  \frametitle{Participant Code of Conduct}
    \relsize{-2}
    \begin{block}{Our Pledge}
      UCAR and NCAR are committed to providing a safe, productive, and welcoming environment for all
      participants in any conference, workshop, field project or project hosted or managed by UCAR, no
      matter what role they play or their background. This includes respectful treatment of everyone
      regardless of gender, gender identity or expression, sexual orientation, disability, physical
      appearance, age, body size, race, religion, national origin, ethnicity, level of experience, political
      affiliation, veteran status, pregnancy, genetic information, as well as any other characteristic
      protected under state or federal law. \href{https://www.ucar.edu/who-we-are/ethics-integrity/codes-conduct/participants}{(\emph{\textcolor{DeepBlue}{link}})}
    \end{block}

    \vspace{2em}
    \textbf{Expected Behaviors}
    \begin{itemize}
    \item All participants are treated with respect and consideration, valuing a diversity of views and opinions
    \item Be considerate, respectful, and collaborative
    \item Communicate openly with respect, critiquing ideas rather than individuals and gracefully accepting criticism
    \item Acknowledging the contributions of others
    \item Avoid personal attacks directed toward other participants
    \item Be mindful of your surroundings and of your fellow participants
    \item Alert UCAR staff and suppliers/vendors if you notice a dangerous situation or someone in distress
    \item Respect the rules and policies of the project and venue
    \end{itemize}

\end{frame}


%%%%%%%%%%%%%%%%%%%%%%%%%%%%%%%%%%%%%%%%%%%%%%%%%%%
\begin{frame}
  \frametitle{A Basic Frame}

  \begin{itemize}
  \item \textbf{Bold bullet}
    \begin{itemize}
    \item Sub-bullet
    \end{itemize}
  \item Enumerated List:
    \begin{enumerate}
    \item item one
    \item item two
    \end{enumerate}
  \end{itemize}
  %% \pause
  \begin{equation}
    \frac{\partial \rho}{\partial t} +
    \frac{\partial}{\partial x_{j}} (\rho u_{j}) = 0
  \end{equation}
  \begin{equation}
    \frac{\partial}{\partial t} (\rho u_{i}) +
    \frac{\partial}{\partial x_{j}} (\rho u_{i} u_{j} + p \delta_{ij} - \tau_{ji} )
    = 0
  \end{equation}
  \begin{equation}
    \frac{\partial}{\partial t} ( \rho e_{0} ) +
    \frac{\partial}{\partial x_{j}}
    ( \rho u_{j} e_{0} + u_{j} p + q_{j} - u_{i} \tau_{ij} ) = 0
  \end{equation}
\end{frame}

\section{Section 1}
%%%%%%%%%%%%%%%%%%%%%%%%%%%%%%%%%%%%%%%%%%%%%%%%%%%
\begin{frame}
  \frametitle{A Frame With a Block}
  \begin{block}{Block Title}
  \begin{itemize}
  \item First bullet
  \item Second bullet
    \begin{itemize}
    \item Sub-bullet
    \end{itemize}
  \item List:
    \begin{enumerate}
    \item item one
    \item item two
    \item item three
    \end{enumerate}
  \end{itemize}
  \end{block}
\end{frame}



\subsection{Subsection}
%%%%%%%%%%%%%%%%%%%%%%%%%%%%%%%%%%%%%%%%%%%%%%%%%%%
\begin{frame}{Columns in beamer}
    \begin{columns}
    \column{0.5\textwidth}
        \centering
        This is column one with 0.5 text width.
    \column{0.4\textwidth}
        \centering
        This is column two with 0.4 text width.
    \end{columns}
\end{frame}



%%%%%%%%%%%%%%%%%%%%%%%%%%%%%%%%%%%%%%%%%%%%%%%%%%%
\begin{NCARprettyframe}[common/images/wallpaper/mesa_lab_sunset.jpg]
  %\pgfsetfillopacity{0.85}
  \vspace{3em}
  \begin{beamercolorbox}[sep=8pt,center]{title}
    \usebeamerfont{title}
    \Huge{\textcolor{HilightGreen}{Questions?}}
  \end{beamercolorbox}
  \begin{exampleblock}{\emph{\textbf{An Example Block, hijacked to make transparent, with optional title}}}
    \centering
    \Huge{\textcolor{UCARGreen}{More Questions?}}
  \end{exampleblock}{}
\end{NCARprettyframe}



\section{Section 2}
%%%%%%%%%%%%%%%%%%%%%%%%%%%%%%%%%%%%%%%%%%%%%%%%%%%
\begin{frame}[t]
  \frametitle{Columns With Blocks Too!}

  \textbf{Flush top [t] option in this frame}

  \begin{columns}[t]
    \column{0.45\textwidth}
    \begin{block}{Left Block Title}
    \begin{itemize}
    \item First bullet
      \begin{itemize}
      \item Sub-bullet
      \end{itemize}
    \item List:
      \begin{enumerate}
      \item item one
      \item item two
      \end{enumerate}
    \end{itemize}
    \end{block}
    \column{0.45\textwidth}
    \begin{block}{Right Block Title}
    \begin{itemize}
    \item First bullet
      \begin{itemize}
      \item Sub-bullet
      \end{itemize}
    \item List:
      \begin{enumerate}
      \item item one
      \item item two
      \item item three
      \item item four
      \end{enumerate}
    \end{itemize}
    \end{block}
  \end{columns}

  \begin{exampleblock}{Example Block Title}
    This should be a restored example block, importantly not messed up with transparency after our \texttt{NCARprettyframe} tomfoolery.
  \end{exampleblock}
\end{frame}



%%%%%%%%%%%%%%%%%%%%%%%%%%%%%%%%%%%%%%%%%%%%%%%%%%%
% Backup
%%%%%%%%%%%%%%%%%%%%%%%%%%%%%%%%%%%%%%%%%%%%%%%%%%%
\begin{NCARprettyframe}[common/images/wallpaper/mesa_lab.jpg]
  %\pgfsetfillopacity{0.85}
  \begin{beamercolorbox}[sep=8pt,center]{title}
    \usebeamerfont{title}
    \Huge{Backup}
  \end{beamercolorbox}
    \begin{exampleblock}{}
      \centering
      \relsize{+4}
      \textbf{Backup}
  \end{exampleblock}{}

\end{NCARprettyframe}



%%%%%%%%%%%%%%%%%%%%%%%%%%%%%%%%%%%%%%%%%%%%%%%%%%%
\begin{frame}{Plain Text}
  \relsize{-1}
  \lipsum[1]
\end{frame}



%%%%%%%%%%%%%%%%%%%%%%%%%%%%%%%%%%%%%%%%%%%%%%%%%%%
\begin{frame}{Quoting Environment}
  \relsize{-1}
  \lipsum[3]
  \begin{quote}
    \lipsum[4]
  \end{quote}
\end{frame}



%%%%%%%%%%%%%%%%%%%%%%%%%%%%%%%%%%%%%%%%%%%%%%%%%%%
\begin{frame}
  \frametitle{Sample Code}

  Here is some sample Python code:
  \lstinputlisting[language=Python]{snippets/hello_world.py}

  Here is some sample C++ code:
  \lstinputlisting[language=C++]{snippets/hello_world.cxx}
\end{frame}



%%%%%%%%%%%%%%%%%%%%%%%%%%%%%%%%%%%%%%%%%%%%%%%%%%%
\begin{frame}[plain,c,shrink]
  \vspace{3em}
  \hspace{1em}\layout{}
\end{frame}



\end{document}
